% Local binhex.tex to satisfy % Local binhex.tex to satisfy % Local binhex.tex to satisfy % Local binhex.tex to satisfy \input{binhex} used by newtxmath.
% Public-domain style helper: defines \binhex{<integer>} -> hexadecimal.
%
% This is a minimal, standalone implementation; you should not need to edit it.

\chardef\binhexA=10
\chardef\binhexB=11
\chardef\binhexC=12
\chardef\binhexD=13
\chardef\binhexE=14
\chardef\binhexF=15

\def\binhex#1{%
  \begingroup
  \count0=#1\relax
  \ifnum\count0<0 %
    \count0=-\count0
    -%
  \fi
  \binhex@digits
  \endgroup
}

\def\binhex@digits{%
  \ifnum\count0>15
    \count2=\count0
    \divide\count2 by 16
    \count1=\count2
    \multiply\count1 by 16
    \advance\count0 by -\count1
    \count0=\count2
    \expandafter\binhex@digits
  \fi
  \binhex@digit
}

\def\binhex@digit{%
  \count2=\count0
  \divide\count2 by 16
  \count1=\count2
  \multiply\count1 by 16
  \advance\count0 by -\count1
  \ifnum\count0<10
    \the\count0
  \else
    \advance\count0 by -10
    \ifcase\count0
      A\or B\or C\or D\or E\or F%
    \fi
  \fi
}
 used by newtxmath.
% Public-domain style helper: defines \binhex{<integer>} -> hexadecimal.
%
% This is a minimal, standalone implementation; you should not need to edit it.

\chardef\binhexA=10
\chardef\binhexB=11
\chardef\binhexC=12
\chardef\binhexD=13
\chardef\binhexE=14
\chardef\binhexF=15

\def\binhex#1{%
  \begingroup
  \count0=#1\relax
  \ifnum\count0<0 %
    \count0=-\count0
    -%
  \fi
  \binhex@digits
  \endgroup
}

\def\binhex@digits{%
  \ifnum\count0>15
    \count2=\count0
    \divide\count2 by 16
    \count1=\count2
    \multiply\count1 by 16
    \advance\count0 by -\count1
    \count0=\count2
    \expandafter\binhex@digits
  \fi
  \binhex@digit
}

\def\binhex@digit{%
  \count2=\count0
  \divide\count2 by 16
  \count1=\count2
  \multiply\count1 by 16
  \advance\count0 by -\count1
  \ifnum\count0<10
    \the\count0
  \else
    \advance\count0 by -10
    \ifcase\count0
      A\or B\or C\or D\or E\or F%
    \fi
  \fi
}
 used by newtxmath.
% Public-domain style helper: defines \binhex{<integer>} -> hexadecimal.
%
% This is a minimal, standalone implementation; you should not need to edit it.

\chardef\binhexA=10
\chardef\binhexB=11
\chardef\binhexC=12
\chardef\binhexD=13
\chardef\binhexE=14
\chardef\binhexF=15

\def\binhex#1{%
  \begingroup
  \count0=#1\relax
  \ifnum\count0<0 %
    \count0=-\count0
    -%
  \fi
  \binhex@digits
  \endgroup
}

\def\binhex@digits{%
  \ifnum\count0>15
    \count2=\count0
    \divide\count2 by 16
    \count1=\count2
    \multiply\count1 by 16
    \advance\count0 by -\count1
    \count0=\count2
    \expandafter\binhex@digits
  \fi
  \binhex@digit
}

\def\binhex@digit{%
  \count2=\count0
  \divide\count2 by 16
  \count1=\count2
  \multiply\count1 by 16
  \advance\count0 by -\count1
  \ifnum\count0<10
    \the\count0
  \else
    \advance\count0 by -10
    \ifcase\count0
      A\or B\or C\or D\or E\or F%
    \fi
  \fi
}
 used by newtxmath.
% Public-domain style helper: defines \binhex{<integer>} -> hexadecimal.
%
% This is a minimal, standalone implementation; you should not need to edit it.

\chardef\binhexA=10
\chardef\binhexB=11
\chardef\binhexC=12
\chardef\binhexD=13
\chardef\binhexE=14
\chardef\binhexF=15

\def\binhex#1{%
  \begingroup
  \count0=#1\relax
  \ifnum\count0<0 %
    \count0=-\count0
    -%
  \fi
  \binhex@digits
  \endgroup
}

\def\binhex@digits{%
  \ifnum\count0>15
    \count2=\count0
    \divide\count2 by 16
    \count1=\count2
    \multiply\count1 by 16
    \advance\count0 by -\count1
    \count0=\count2
    \expandafter\binhex@digits
  \fi
  \binhex@digit
}

\def\binhex@digit{%
  \count2=\count0
  \divide\count2 by 16
  \count1=\count2
  \multiply\count1 by 16
  \advance\count0 by -\count1
  \ifnum\count0<10
    \the\count0
  \else
    \advance\count0 by -10
    \ifcase\count0
      A\or B\or C\or D\or E\or F%
    \fi
  \fi
}
